\documentclass[a4paper]{article}
\usepackage[utf8]{inputenc}
\usepackage[T1]{fontenc}
\usepackage{hyperref}
\usepackage[francais]{babel}
\author{Georges Ricci\\
\large \textit{X-Alternative}}
\date{\today}
\title{Le développement de la Chine\\ 
 \large \textit{Une stratégie de très longue période}
 }
\renewcommand{\contentsname}{Table des matières}
\begin{document}

\maketitle 

\newpage
\tableofcontents
\newpage

L’émergence actuelle de la Chine dans le domaine techno-scientifique est particulièrement frappante. Elle est la résultante d’une stratégie impulsée par le Parti Communiste Chinois (PCC), sur une très longue période.

\section{Le plan le plus récent: Made in China 2025}
\label{sec:org5915e2c}
\subsection{Tour d’horizon rapide}
\label{sec:org35ecd62}
Le plan \textit{« Made in China 2025 »} a été adopté en 2015, sous le patronage du premier ministre chinois \textit{Li Keqiang}, et de son cabinet, le \textit{Conseil des affaires de l'État de la république populaire de Chine}. Il est décrit en détail dans \cite{Made_In_China_2025}. 

Le calendrier de ce plan s’étire désormais jusqu’en 2049, où la Chine devrait être devenue une puissance industrielle leader. \cite{Evolving_MiC25} 

Ce plan identifie dix domaines clés, prioritaires à développer, afin de permettre à la Chine de mettre à niveau son industrie, et de devenir autonome sur certaines technologies critiques.

Ce sont~:
\begin{itemize}
\item Les nouvelles technologies de l’information;
\item L’aviation et l’espace;
\item Les équipements maritimes;
\item Les équipements du secteur énergétiques;
\item Les machines agricoles;
\item Les nouveaux matériaux;
\item La biopharmacie et les dispositifs médicaux de haute technologie;
\item Les machines-outils de précision et la robotique.
\end{itemize}

\subsection{Une volonté Chinoise de reprise en main de sa souveraineté}
\label{sec:orgf222c4b}
La Chine a cherché à réduire sa dépendance à l’égard de l’étranger, mais aussi à creuser l’écart avec d’autres nations en voie de décollage, qui s’avèrent être capable de remonter, elles aussi, les chaînes de valeurs. 

Au coeur de la stratégie \textit{« Made in China 2025 »} est le contrôle des technologies pivots des processus industriels, l’indigénisation des chaînes de productions et logistiques, mais l’acquisition de compétences stratégiques, et d’innovation.

Ce plan se fixait l’objectif que « 70\% des fournisseurs de composantes électroniques de base du marché intérieur chinois » soient autochtones.

Un des premiers objectif à atteindre dans le domaine des semiconducteurs, était d’assurer qu’en 2021, 25\% de ceux ci, sur le marché intérieur, proviennent de fabriquants chinois \cite{Made_In_China_2025}.

En réalité, en 2021, cette part n’a été que de 10\% \cite{Futur_China_Semiconductor}.

C’est donc toute une grappe de technologies et de savoir-faires de base, comme le montre \cite{Made_In_China_2025}, nécessaire à sa souveraineté, que la Chine désirait aquérir à son lancement.

\subsection{Le rôle de l’état chinois}
\label{sec:org30c6878}
\subsubsection{6 modes d’interventions de l’état chinois}
\label{sec:orgecd012a}
L’état chinois, actuellement, peut user de six voies différentes d’intervention~:
\begin{itemize}
\item L’état chinois est directement propriétaire d’une entreprise;
\item Les managers d’une entreprise, peuvent être des cadres du PCC. Il y a ainsi un
flux constant de ses cadres, passant du gouvernement à la direction
d’entreprises. Cela assure lui une connaissance précise des rouages de
l’économie chinoise;
\item Les prêts du système financiers sont contrôlés par l’état chinois;
\item L’état chinois définit une politique industrielle, au moyen de plans pluriannuels;
\item L’état est un acheteur essentiel, qui se fournit auprès de ses entreprises
d’états, leur assurant ainsi un débouché stable et structurant;
\item L’état contrôle des agences de régulation, dont les actions sont souvent
politiquement orientées.
\end{itemize}

On pourra consulter, pour plus de détails, le chapitre 4 de \cite{heilmann17_chinas}. 

\subsubsection{« \textit{Innovation-driven Development Strategy} » (idds)}
\label{sec:org3a525f5}
    Le plan \textit{« Made in China 2025 »} s’intègre dans une stratégie bien plus vaste, nommée en anglais, \textit{« Innovation-driven Development Strategy » (idds)}, lancée à partir de 2015-2016. Il s’agit d’une évolution d’un programme de 2006, par \textit{Wen Jibao}, assurant un retour de l’état chinois dans la formation des stratégies industrielle. De nouvelles structures de financement ont été alors créé.


\subsubsection{Les « \textit{Industrial Guidance Funds} »}
\label{sec:org4650fde}
« \textit{Made in China 2025} » tente de fusionner à la fois des mécanismes de marché, mais aussi un pilotage par l’état Chinois.

Il faut cependant avoir conscience que, désormais, la notion de « pilotage » a un sens assez spécifique en Chine. En effet, on doit comprendre cela comme un pilotage indirect par l’état, qui s’exerce au moyen des « \textit{Industrial Guidance Funds} » (IGF), ayant atteint 11,27 mille milliards de renminbi en 2018. 

Chaque fonds a un objectif d’investissement, défini de façon assez large et flexible. Les investissements sont centralisés et massifs, afin d’arriver rapidement à des résultats, et en s’attaquant à tout les niveaux d’une chaîne de production par exemple. De fait, l’état chinois accorde sa garantie à ces fonds, ce qui ne leur incitent pas à rechercher un retour rapide sur investissement, ni à des performances financières très grandes.

L’inconvénient majeur est le développement d’une corruption basée sur l’ac\-cès à ces sources de financement, et à la diffusion systémique aux entités éco\-no\-mi\-ques participant à ces fonds, des risques sous-jacents de ces investissements. Voir pour cela la discussion dans \cite{ang20_chinas} de la notion de \textit{corruption d’accès} et de son rôle ambigü dans la prospérité chinoise).

 Ces \textit{IGF} sont essentiellement contrôlés au niveau local, par les gouvernements provinciaux, ce qui leur permet d’êtres des relais très rapides des politiques décidées au centre de l’état chinois.

 C’est ainsi que le PCC a régulièrement recours a un mode de mobilisation de ses forces, particulièrement ancré dans son origine révolutionnaire, que Sebastian Heilmann dans \cite{heilmann18_red} a appelé le « \textit{guerrilla policy style} ».

Il s’agit d’un mécanisme en avalanche, profitant des moindres opportunitées sur le terrain, encourageant les expérimentations locales, et les « bricolages » informels, tout en tolérant les gaspillages dues à leurs échecs. 

\subsubsection{Le rôle des entreprises d’États}
\label{sec:org5c8608f}

Un autre canal d’intervention de l’état chinois est la subvention directe des des entreprises d’état, les « \textit{State-Owned Enterprises} » (SOE).  

On peut en distinguer trois types~:
\begin{itemize}
\item Les entreprises d’état stratégiques pour l’état chinois, qui dominent largement leur branche, et qui sont des monopoles ou des oligopoles;
\item Celles de tailles moyennes, détenues par des gouvernements locaux, dans des secteurs non stratégiques, et soumisent à une concurrence intense;
\item Des entreprises en apparence privées, mais ayant de forts liens avec l’état chinois, comme par exemple \textit{Huawei}, ou bénéficiant de protection particulière de l’état, car assurant pour lui des services, comme le fait \textit{Tencent} par exemple pour \textit{Le Grand Parefeu}\footnote{Sur le rôle du \textit{Grand Parefeu}, comme barrière non tarifaire ayant permis l’émergence des géants chinois de l’internet, on pourra consulter \cite{griffiths19_china}}.
\end{itemize}


Bien que le secteur privé emploie de l’ordre de 70\% de la force de travail chinoise \cite{xiaolan15_chinas}, les SOEs, du fait leurs tailles et des liens privilégiés qu’elles entretiennent avec l’état chinois et ces hauts fonctionnaires, servent ainsi de relais des politiques voulues par le PCC.

Malgré une capacité moindre à innover, ces liens leur permettent de profiter des crédits d’un système financier totalement contrôlé par le PCC, et de réorganiser autour d’elles la structure de marché d’une branche industrielle\footnote{Lire pour cela \cite{naughton21_rise}}.

Dans le cadre de « \textit{Made in China 2025} », les SOEs représentent 83\% des revenues des secteurs industriels structurels que ce plan renforce \cite{Evolving_MiC25}.

Ces entreprises sont d’ailleurs les principaux partenaires des fonds d’in\-vestis\-sements précédemement évoqués. Bien que pesant moins que par le passé dans l’économie chinoise, ces entreprises continuent d’avoir un rôle important de redistribution sociale~: elles assurent ainsi une part du bouclage macro-économique que les systèmes sociaux occidentaux réalisent.

Elles permettent alors de recycler les profits réalisées lorsque les plans d’in\-vestis\-sement aboutissent. Ces structures assurent en pratique, bien que de façon non démocratiques et sans être exemptes de biais ni de corruption, la stratégie proposée par Mariana Mazzucato dans \cite{mazzucato18}.

Elles réalisent ainsi une socialisation par l’état chinois des gains réalisés lors d’investissements risqués réussis, et lui permettent d’accroître ses compétences techniques et managériales pour le pilotage de l’économie chinoise.

\subsection{Résoudre des faiblesses structurelles.}
\label{sec:orgebd142e}
« \textit{Made in China 2025} » tente de résoudre des problèmes qui deviennent critiques pour le PCC.

\subsubsection{Une dépendance structurelle à l’étranger}
\label{sec:orgd34c67b}
L’un des premiers de la Chine est sa dépendance aux technologies étrangères et à son positionnement dans les chaînes de valeurs. Elle subit ainsi un « effet sandwich »~: d’un côté, elle est mise en compétition avec des nations reproduisant son parcours industriel, imitant ses technologies et érodant ses avantages comparatifs. De l’autre, elle subit la pression de nations ayant de plus grandes capacités d’innovations, qui extraient des rentes de monopôles sur des brevets essentiels, ou même en bloquant selon leur bon vouloir l’acquisition de technologies clés\footnote{(Lire ainsi \cite{dollar20_china}}.

La Chine est ainsi contrainte d’importer pour plus de 300 milliards de dollars de produits étrangers de haut-technologie pour assembler ses propres produits\footnote{Voir pour cela le graphique p 24 de \cite{Evolving_MiC25}}.

La dépendance à l’étranger ne se limite pas d’ailleurs à cela~: en effet, ne pouvant compter sur un environnement local suffisamment propice, la Chine dépends fortement de sa participation à des collaborations internationales pour l’innovation, et au retour de membres de sa diaspora pour acquérir de nouvelles compétences\footnote{Lire la discussion dans \cite{xiaolan15_chinas}}.

La Chine, grâce à \textit{Made in China 2025} essaie de s’échapper du « piège du revenu intermédiaire », et d’acquérir des compétences, notamment des savoirs tacites, par des programmes d’acquisition d’entreprises étrangères, et par le « learning by doing » local.

\subsubsection{Faiblesse de l’innovation}
\label{sec:orga58cd43}
Un trait saillant de l’économie chinoise est la faiblesse globale de l’innovation, cachée par quelques secteurs, notamment dans la haute technologie très dynamiques. Comme le montre \textit{Xiaolan Fu} dans \cite{xiaolan15_chinas}, le seul secteur de la haute technologie compte pour 75\% des brevets en Chine.

L’autre caractéristique chinoise est la faiblesse générale des investissements dans la recherche de base. La Chine était jusqu’en 2011 l’un des pays investissant le moins dans ce domaine par rapport à sa catégorie économique\footnote{Voir \cite{xiaolan15_chinas}, et lire l’appendice de \cite{simon09_chinas}}. Malgré de très grand investissements, remontant aux plans quinquenaux des années 80, les publications scientifiques chinoises, dont le nombre a doublé entre 2005 et 2010 sont restées peu citées à l’échelle internationale.

En 2006 a ainsi été lancé le « \textit{Medium- to Long-Term Plan for the Development of Science and Technology} » (MLP), un plan de 15 ans, se fondant dans le plan de 2006 lancé par \textit{Wen Jibao} \cite{cao06_chinas_scien_techn_plan}, qui était destiné à limiter la dépendance de la Chine à un niveau de pas plus de 30\% de technologies importées. Il en a résulté une forte augmentation du personnel de R\&D, ne le résolvant pas cette question cependant, puisque, parallèlement, les besoins en personnels très qualifiés se sont accrus dans l’industrie, provoquant un départ vers le secteurs privé ou vers les entreprises étrangères installées en Chine \cite{simon09_chinas}.

\subsubsection{La question démographique}
\label{sec:org2d752fd}
Le choix de lancer « \textit{Made in China 2025} » a aussi une origine démographique.  L’un des points clé de cette décision est le constat du vieillissement de la population chinois, du à la politique de l’enfant-unique mise en oeuvre initialement sous Deng Xiaoping, en 1979, puis considérablement assouplie en 2015\footnote{Pour une histoire du processus menant à ce choix drastique, lire \cite{greenhalgh08_just}}.

La Chine, du fait cette politique s’est ainsi rapproché du « \textit{point de bascule de Lewis} », c’est à dire de l’instant démographique où le stock de main-d’oeuvre à bas coût, suite au vieillissement de la population et à son urbanisation, n’est plus suffisant pour empêcher l’élévation du coût de la main d’oeuvre\footnote{Voir \cite{dollar20_china}}.

L’avantage compétitif du coût de la main-d’oeuvre chinoise s’érode donc. Une nouvelle thématique a ainsi donc émergé, celle de la \textit{qualité}, et plus généralement, celle de la \textit{qualité de la population chinoise}, se cristallisant autour du concept de \textit{Suzhi}.

Cette notion s’est trouvé inscrite dans le 17ème plan quinquénal dès 2004. 

Elle mixe ensemble des notions provenant du Confucianisme, du Maoïsme, du Darwinisme Sociale, et du Néolibéralisme, encourageant les incitations à l’amélioration du capital humain chinois\footnote{Voir \cite{greenhalgh10_cultiv}}.

Cela s’est traduit par des objectifs de croissances du nombre de diplômés dans le domaine des STEM, mais aussi d’élever le niveau global de la population à un niveau scolaire équivalent à celui de la fin du cycle secondaire. Cependant, il en résulte un accroissement des inégalités d’accès, par un renforcement de la compétition scolaire. Cela creuse radicalement le gouffre entre les populations citadines et paysannes. Le niveau moyen de la population reste cependant en dessous des objectifs du PCC, créant des pénuries de main-d’oeuvre qualifiées\footnote{Voir \cite{simon09_chinas} et \cite{xiaolan15_chinas}}.


\section{Une imbrication de niveaux d’interventions}
\label{sec:org0a2f1cf}
La brève présentation précédente de \textit{« Made in China 2025 »} a permis de constater la plasticité du système chinois. Comme on a pu le voir, il repose sur deux grandes caractéristiques~:

\begin{itemize}
\item des engagements sur le temps long, mobilisateurs, portant sur des aspects structuraux de l’économie, mais aussi de la société chinoise;
\item une capacité à expérimenter et à prendre des risques, afin que les différentsacteurs apprennent par leur erreurs, mais aussi par leur succès.
\end{itemize}

\subsection{Expérimentation et planification}
\label{sec:org4828e82}
\subsubsection{La planification}
\label{sec:org7c0e0d7}
Depuis la période de la Réforme, post 79, on peut distinguer 3 grandes évolution de la notion de planification en Chine \cite{heilmann18_red}.

\begin{itemize}
\item Entre 1980 et 1992, la planification chinoise vise la croissance des biens matériels, repose sur un guidage administratif, gère les ressources via des ministères de branches;
\item Entre 1993 et 2004, son objectif est d’assurer un contrôle macro-éco\-no\-mi\-que, afin d’arriver à une transition vers une croissance plus soutenable. Les différentes branches d’industries et leur ministères sont démantelées;
\item Après 2004, la planification s’occupe de restructurer des secteurs entiers, de redistribuer les gains réalisés, dans une perspective de stabilité sociales. Et elle se caractérise par un grand nombre programmes multi-années sectoriaux.
\end{itemize}
\subsubsection{L’expérimentation}
\label{sec:orgcd915f5}
L’innovation politique majeure du PCC est son usage de l’expérimentation dans le domaine des politiques économiques. Ainsi entre 2003 et 2006, il y a pas moins de 138 programmes expérimentaux, coordonnant 31 conseils, de niveaux ministeriels, dans des domaines aussi différents que l’économie rurale, la régulation financière, ou l’éducation. Chaque expérimentation a lieu simultanément en plusieurs lieux.

Cela s’est traduit par la création de \textit{Zones Expérimentales}, où le PCC teste ces nouvelles politiques, les évalue puis peut décider d’étendre à plus vastes échelles ces réformes. Les politiques sont ainsi discutées, évaluées, sous la forme d’une boucle de rétroaction entre le niveau national et le niveau local.

Les différents plans, conçus au niveau central, sont alors dupliqués, expérimentés, amendés à l’échelle régionale. Nombre d’entre eux d’ailleurs s’ap\-pli\-quent à une échelle plus vaste, notamment pour les infrastructures pour résoudre des goulots d’étranglement. Cela conduit à l’association de plusieures régions, souvent avec des fortes disparités économiques, qui permettent de créer des synergies, l’une fournissant la main d’oeuvre pour l’autre. C’est ainsi qu’à l’échelle interne, le PCC applique la stratégie du « vol d’oies » japonaise entre ses différentes régions.

\subsection{Le niveau mésoscopique.}
\label{sec:orgc8b3031}
L’application des différents plans reposent sur l’adaptation de leur objectifs plutôt généraux, par des haut-fonctionnaires, dont la carrière et le rang dépends de leurs succès. Ces derniers sont définis de façon assez étroites et explicites par le PCC. L’échec cependant amène essentiellement un ralentissement de leur carrière, ce qui laisse donc une possibilité d’apprentissage.

Si cette latitude dans l’adaptation des plans centraux permet aux differents strates de la bureaucratie chinoise de comprendre les besoins et les contraintes locales, les différentes incitations à réaliser de la croissance, elles, peuvent amener aux gaspillages de ressources, et un développement de la corruption \cite{ang16_how_china}.

\subsection{L’essort de la corruption}
\label{sec:orga8aaf5f}
La montée de la corruption a débuté dans les années 90, après le massacre de Tiananmen, suites aux réformes libérales lancées par le gouvernement chinois. L’origine de ses réformes fut le voyage dans le Sud de Deng XiaoPing en 1992~: les réformes économiques s'accélèrèrent, et furent constituées de privatisations pleines ou partielles, dénommées par euphémisme « réformes des droits de propriétés ».

Elles aboutirent à faire passer la part de l’état dans le PNB chinois de 35\% en 1979 à 11\% en 1996. Malgré une forte discipline fiscale, il en résulta un effondrement des recettes de l’état, mais aussi de la redistributions aux régions les plus pauvres \cite{wong09_rebuil_gover_centur}.

De fait, les inégalités entre régions se sont creusées et le système fiscal intergouvernemental s’en est retrouvé brisé. Cette décentralisation de fait a poussé les régions, pour réaliser les objectifs définis par le centre, notamment en terme de fournitures de bien public, à faire reposer leur budgets sur des activités extra-budgétaires. En 1994, face au désordre induit, un nouveau système a été créé~: le Système de Partage des Taxes, afin de renverser la tendance à la décentralisation.

Mais, en supprimant la possibilité pour les régions et les communes de lever leur propre impôts, elle a accentué leur dépendance à l’égard des sources extra-budgétaires \cite{wong09_rebuil_gover_centur}.

Or les réformes de libéralisation ont crée un flou entre « les droits de propriétés », qui reste ceux de l’état, et les « droits d’usage », qui sont ceux du Parti Communiste et donc de ces membres. Cela a donné un grand pouvoir aux officiels, et aux managers des entreprises d’états pour le contrôle de ces droits et leur cession \cite{pei16_chinas}.

Il en a ainsi résulté un marché, alimentant les budgets des collectivités et des régions, permettant le financement des plans régionaux adaptés des plans centraux mais aussi des accaparements et des enrichissements personnels illégaux.

La question de la corruption en Chine reste assez subtile, car on ne peut pas totalement l’aborder en terme d’inefficacité. En effet, du fait de ressources fiscales assez faibles pour l’état chinois, et du fait des insuffisances des mécanismes de redistributions entre l’état central et les régions, la mise en oeuvre des différents plans, ne peut se faire qu’en ayant recours à des « bricolages » financiers, notamment par l’accaparement de terre, et leur utilisation comme collatéral dans différents produits financiers. De même, la difficulté pour le secteur privé d’accéder aux crédits, et la contrainte qu’ont les bureaucrates de les associer à leur plan de développement amènent à développer une corruption dont le but est l’accès privilégié au crédit et aux informations liées au développement des différents plans \cite{ang20_chinas}. 

\section{Avant la réforme de 1979}
\label{sec:org112d285}
\subsection{L’impact de la guerre de Corée}
\label{sec:org77dd9e3}
La période pré-Réforme de 1979 n’a pas cependant été une période perdue. Une part essentielle du coeur des compétences technologique de la Chine résulte en effet des grands quinquennaux découlant du retours d’expérience sur l’intervention chinoise dans la Guerre de Corée.

Il s’en est suivi un programme d’armements de haute technologie, destiné à doter la Chine de l’ensemble des moyens nécessaire à une stratégie de dissuation. L’acquisition des compétences nécessaires a résulté de l’aide soviétique, mais aussi le retour d’universitaires de la diaspora chinoise, s’étant exilé aux États-Unis et en Europe, lors de l’occupation japonaise, puis la guerre civile.

Ainsi, le père de l’astronautique chinoise, Qian Xuesen\footnote{voir \cite{chang95_thread}, pour une biographie}, après avoir été formé aux États-Unis dans les années 30, accusé d’être communiste reviendra en Chine en 1955 lors d’un échange de prisonniers. Il deviendra un conseiller extrêmement influent de Zhou Enlai, définissant ainsi les orientations du plan de 1956 de développement scientifique.

Un autre aspect a été le développement d’un gigantesque programme d’infra\-structures duales, au travers tout le pays, dispersant l’infrastructure militaro-industrielle critique. Cela contribua à diffuser les bases industrielles, mais aussi les compétences et les savoir-faires techniques dans des régions comme le Sichuan, qui sont désormais devenue prioritaires dans le cadre du plan de développement de 2006.

L’ampleur de ce \textit{Troisième Front} commencent à peine à émerger aux yeux des historiens occidentaux, mais il permet de comprendre que le développement actuelle de ces régions éloignées des côtes s’appuie sur une continuité d’efforts\footnote{Lire à ce sujet \cite{meyskens20_maos}}.


\subsection{Le plan de développement des sciences et des technologie de 1956}
\label{sec:org13d69c0}
En avril 1956, une conférence réunissant notamment le Maréchal Nie et Qian Xuesen, sous la direction de Zhou Enlai pose les bases du \textit{Plan de Douze Ans} pour le développement des sciences et des technologies. Il définissait 57 domaines prioritaires, et notamment l’informatique, l’électronique ou l’aérospatial. La protection politique de Zhou Enlai, et surtout l’habileté du Maréchal Nie permirent alors une large continuité de l’attribution de ressources à cet ambitieux programme de développement \cite{feigenbaum03_chinas}. 

Dès ce plan, on retrouve le style d’organisation des plans ultérieurs~:
\begin{itemize}
\item une mobilisation flexible des moyens;
\item l’identification par l’état de technologies critiques, qui ont une implication pour la position relative de la Chine dans l’équilibre des puissances mondiales;
\item un investissement massif de l’état dans ces secteurs;
\item un programme de substitution des technologies par une production locale;
\item la création d’une capacité interne d’innovation dans ce domaine;
\item une diffusion de cette technologie du domaine militaire au domaine civile en dernier étape.
\end{itemize}

Bien que très verticale dans son organisation, dans la plupart des branches, celle du nucléaire et des missiles ballistiques sera bien plus horizontale dans son organisation. Après le chaos de la Révolution Culturelle, les ingénieurs et généraux seront les artisans de la reconstruction du système de recherche chinois \cite{feigenbaum03_chinas}.

Notons, cependant, à la suite de \cite{andreas09_rise} que malgré le chaos résultat des luttes entre factions de Gardes Rouges au sein de Tsinghua, ce centre d’élite continua de jouer son rôle de formation, et permis même une forme d’égalitarisme et de diffusion hors de ce groupe social, de technologies avancées, comme l’in\-for\-ma\-ti\-que.

\section{Conclusion}
\label{sec:org522d513}
Ce rapide tour d’horizon permet de prendre conscience de l’ampleur des efforts sous-jacents à l’émergence actuelle de la Chine.

Ils permettent aussi de relativiser leur rapidité~: c’est la conséquence d’une longue série d’étapes, d’impasses, d’échecs, mais aussi d’une volonté politique d’une grande constance.

On ne peut comprendre ces efforts si l’on oublie la longue période débutant avec les Guerres de l’Opium, perdues par la Chine, qui ont profondément traumatisé les élites chinoises.

Les différents plans de modernisation chinois s’ancrent dans une double peur du chaos.

L’une est à l’égard de l’extérieur, et détermine sa Grande Stratégie de recherche d’autonomie. Elle détermine sa volonté de bâtir un espace géopolitique stable, détournant les flux de matière vers elle, le structurant par la finance mais aussi aussi, à terme, par ses forces armées \cite{khan18_haunt}. 

L’autre est celle du chaos intérieur. Si la classe moyenne urbaine supporte le PCC, malgré sa corruption, ses échecs et ses mensonges, c’est que ce dernier continue d’assurer une croissance quantitative et de plus en plus qualitative, \cite{dickson16}.

Cependant, au fur et à mesure que cette classe moyenne a des aspirations croissantes en terme de qualité de services apportés par l’état, les contradictions dues au mode de fonctionnement du système chinois, ne cessent de s’accroitre, comme le montrent \cite{cabestan18_demain_chine} et \cite{dickson16}.

 Le PCC est en effet piégé dans un ensemble de doubles contraintes de plus en plus complexes à démailler. Comment assurer par exemple un environnement acceptable pour les universitaires chinois et expatriés occidentaux, dont la Chine a besoin pour améliorer sa capacités d’innovation, tout en renforçant son contrôle sur la production intellectuelle, si l’on suit \cite{haour16_creat_china}
et \cite{simon09_chinas}?

Comment assurer la confiance entre les individus quand le fonctionnement opaque du PCC encourage la formation de cabales et empêche l’existence d’orga\-ni\-sa\-tions indépendantes~?

Par bien des aspects, la vie quotidienne des classes moyennes et populaires chinoise reste difficile~: le système de santé chinois est très inégalitaire, l’éducation tout autant et de plus en plus compétitive dans la logique de la politique du \textit{Suzhi} \cite{damien14_in}.

Le PCC est donc pris dans une course visant tout à la fois à maintenir sa domination, et les rentes qu’elles procurent, mais aussi à empêcher que l’anomie ne se répande en son sein, en cas de défaillance de son modèle économique.

À grande échelle, la résolution des déséquilibres globaux entre Chine, États-Unis et Europe, qui traversent les classes sociales de ces pays en devient de plus en plus complexes.

L’absence d’un filet de sécurité social conséquent, la persistence d’une répression financière empêchant les investissements transparents des particuliers crée une immense épargne, s’investissant dans des capacités de production de plus en plus excédentaires, suivant des décisions souvent opaques, accroissant les déséquilibres mondiaux déstabilisant comme le montre Michaël Pettis dans \cite{klein20_trade}. 

Un pays moyen comme la France est donc pris dans les remous que provoquent l’émergence Chinoise, et notamment la compétition croissante entre les États-Unis et la Chine pour l’hégémon global. La France ne peut donc ignorer la Chine. Elle ne peut pas non plus espérer en secret un échec chinois, qui de toute façon aurait aussi des conséquences négatives sur elle.

Elle peut cependant réagir, en s’appuyant sur ses capacités qui ne sont pas totalement épuisées, lister ce dont elle a besoin, ce qu’elle a encore. Bref, d’une certaine façon, imiter ce qu’a fait la Chine, en bien plus vaste, car en partant de bien plus bas.


\section{Bibliographie commentée}
\label{sec:orgc4508dc}

Les rapports \cite{Made_In_China_2025} et \cite{Evolving_MiC25} décrivent en grand détails \textit{« Made in China 2025 »}, son organisation et ses faiblesses.

\cite{naughton21_rise} est un petit ouvrage exposant à la fois l’histoire et l’évolution de la planification, et de la politique industrielle en Chine depuis la réforme. Son ouvrage \cite{naughton18_chines} est un panorama de l’état actuel de l’économie et des institutions chinoise.

\cite{heilmann18_red} est une intéressante présentation de la stratégie économique \textit{non conventionnelle} du PCC. Plusieurs chapitres sont consacrés à la planification en Chine et à son évolution. Les deux ouvrages de Yuen Yuen Ang, \cite{ang20_chinas} et \cite{ang16_how_china} forment un dyptique théorique, dans l’esprit de l’économie des institutions, sur la renaissance de l’institution du marché en Chine. Elle est cependant assez biaisée de ce fait, et n’aborde pas la question de la politique industrielle et de son lien avec la planification. Son analyse du rôle de la corruptione en Chine est cependant très éclairante. 

\cite{feigenbaum03_chinas} est une histoire du complexe techno-militaire chinois et de son importance pour la politique industrielle longue, mais aussi sur l’enseignement supérieur et la recherche chinoise.

\cite{heilmann17_chinas,dickson16} et \cite{cabestan18_demain_chine} présentent les institutions chinoises et les défis auxquels elles vont devoir répondre. 

\cite{damien14_in} est une synthèse accessible et complète des nombreux goulots d’étran\-gle\-ment affectant la vie quotidienne des classes moyennes et populaires chinoises. Une synthèse plus austère et bien plus précise, mais avec un fort biais néolibéral est \cite{dollar20_china}.

\cite{xiaolan15_chinas,haour16_creat_china,hong17_networ_china} étudient chacun un aspect des enjeux liés à l’innovation en Chine.

Pour une présentation très complètes de l’enseignement supérieur chinois et de son histoire, on peut consulter \cite{andreas09_rise} et \cite{simon09_chinas}.

Pour une vision globale des enjeux de l’émergence de la Chine, on peut lire \cite{shambaugh13_china}, et pour une vue d’aigle des déséquilibres économiques et de la place qu’y occupe la Chine, \cite{klein20_trade} est un excellent point de départ. Sur la Grande Stratégie chinoise, \cite{khan18_haunt} est une introduction rapide et efficace.

La question de la démographie est étudié en grand détail, suivant une approche anthropologique dans \cite{greenhalgh08_just,greenhalgh05_gover_chinas} et \cite{greenhalgh10_cultiv}.

\bibliographystyle{ieeetr}
\bibliography{La_Strategie_Developpement_Chinoise}
\end{document}
